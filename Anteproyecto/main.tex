\documentclass[11pt, spanish, letterpage]{article} % {{{
\usepackage[T1]{fontenc}
\usepackage{tikz}
\usetikzlibrary{quantikz}
\usepackage[utf8]{inputenc}
\usepackage[letterpaper]{geometry}
\geometry{verbose,tmargin=2cm,bmargin=2.5cm,lmargin=2cm,rmargin=2cm}
%\pagestyle{plain}
\setlength{\parskip}{\baselineskip}%espacio entre parrafos
\setlength{\parindent}{0mm}
\usepackage{graphicx}
%\usepackage{setspace}
\usepackage{tabulary}
\usepackage{siunitx}
\usepackage{amsmath}
\usepackage{amsfonts}
\usepackage{amssymb}
\usepackage{amsthm}
\usepackage{physics}
\usepackage{wrapfig}% para colocar figuras en diferentes posiciones
\usepackage{bbold}

\usepackage{fancybox}
\usepackage{colortbl}
\usepackage{amsbsy}
\usepackage[draft,inline,nomargin]{fixme} \fxsetup{theme=color}
\FXRegisterAuthor{cp}{acp}{\color{blue}CP}
\FXRegisterAuthor{ja}{aja}{\color{orange}JA}

\usepackage{lipsum}
\usepackage{babel}
\usepackage{multirow}
\usepackage{array}

\usepackage[]{lineno}  %\linenumbers
%\setlength\linenumbersep{3pt}

\renewcommand{\baselinestretch}{1} % interlinado
\addto\shorthandsspanish{\spanishdeactivate{~<>}}
\date{}
\spanishdecimal{.}
\usepackage{multicol}%para escribir en muchas columnas
%para que no corte palabras
\usepackage[none]{hyphenat}
\usepackage{times}
%\onehalfspacing

\usepackage{hyperref}
%\usepackage{biblatex}

%---ejercicios, problemas -teoremas
%---Problemas encerrados-Bonitos
\usepackage[framemethod=tikz]{mdframed}
\mdfsetup{skipabove=\topskip,skipbelow=\topskip}
\newcounter{problem}[section]
\newenvironment{problem}[1][]{%
%\stepcounter{problem}%
\ifstrempty{#1}%
{\mdfsetup{%
frametitle={%
	\tikz[baseline=(current bounding box.east),outer sep=0pt]
	\node[anchor=east,rectangle,fill=brown!50]
{\strut Problema~\theproblem};}}

}%
{\mdfsetup{%
frametitle={%
	\tikz[baseline=(current bounding box.east),outer sep=0pt]
	\node[anchor=east,rectangle,fill=brown!50]
{\strut Problema ~\theproblem:~#1};}}%

}%
\mdfsetup{innertopmargin=5pt,linecolor=black!50,%
	linewidth=2pt,topline=true,
	frametitleaboveskip=\dimexpr-\ht\strutbox\relax,}
	
\begin{mdframed}[]\relax%
}{\end{mdframed}}

\newenvironment{solution}% environment name
{\colorbox{gray}{~~\textbf{\textcolor{white}{Solución:}}~~}~~}%
{}
%-----end------------

%\newtheorem{example}{Ejemplo}[chapter]
%\newtheorem{ejercicio}{Ejercicio}[chapter]
%%---
\newcommand{\Ev}{\mathbf{E}}
\newcommand{\rv}{\mathbf{r}}
\newcommand{\ru}{\hat{\rv}}
\usepackage{tabulary}
%---paquetes para fisica
\usepackage{physics}%facilita la escritura de operadores usados en fisica
%-paquete para unidades en el sistema internacional
% \usepackage[load=prefix, load=abbr, load=physical]{siunitx}
% --definiendo constantes fisicas en el SI
% \newcommand{\accgravity}{9.8 \metre / \Sec^2}
% ---diferencial inexacta
% \newcommand{\dbar}{\mathchar'26\mkern-12mu d}

\oddsidemargin 0in
\textwidth 6.5in
\topmargin -0.5in
\textheight 8.5in
% }}}
\begin{document}
\begin{titlepage} % {{{ Suppresses displaying the page number on the title page and the subsequent page counts as page 1                                  
\newcommand{\HRule}{\rule{\linewidth}{0.5mm}} % Defines a new command for horizontal lines, change thickness here                             

\center % Centre everything on the page                                                                                                       

%------------------------------------------------                                                                                             
%       Title                                                                                                                                 
%------------------------------------------------                                                                                             
	
\HRule\\[0.6cm]

{\huge\bfseries Implementación de canales cuánticos de un qubit usando Algoritmos Variacionales Cuánticos (VQA) }\\[0.5cm] % Title of your document                                                 

\HRule\\[2cm]

%------------------------------------------------                                                                                             
%       Author(s)                                                                                                                             
%------------------------------------------------                                                                                             


\Large{\textbf{Felipe Antonio Ixcamparic Choy}}\\ [2cm] % Your name                                                                                          

%------------------------------------------------                                                                                             
%       Headings                                                                                                                              
%------------------------------------------------                                                                                             

\textsc{\LARGE Universidad de San Carlos de Guatemala\\ Escuela de Ciencias Físicas y Matemáticas\\ Licenciatura en Física}\\[2cm]
\textsc{\huge Anteproyecto}\\
\textsc{\Large Ejercicio Profesional Supervisado}\\[2cm]

\textsc{\Large Supervisado por: \textbf{Dr. Carlos Francisco Pineda Zorrilla (IF-UNAM)} \\ y  \textbf{Dr. Giovanni Ramirez (ECFM-USAC)} }  %	} 
                                                                                                      

%------------------------------------------------                                                                                             
%       Date                                                                                                                                  
%------------------------------------------------                                                                                             
\vfill\vfill\vfill % Position the date 3/4 down the remaining page
\vfill\vfill\vfill

{\large \today} % Date, change the \today to a set date if you want to be precise                                                              

%------------------------------------------------                                                                                             
%       Logo                                                                                                                                  
%------------------------------------------------                                                                                             

%\vfill\vfill                                                                                                                                 
\includegraphics[width=0.15\textwidth]{Imagenes/0001.jpg}\\[1cm] % Include a department/university logo - this will require the graphicx packag\                                                                                                                                                  

%----------------------------------------------------------------------------------------                                                     

\vfill % Push the date up 1/4 of the remaining page      
\end{titlepage} % }}}

%\section{Descripción general de la institución}

La Universidad de San Carlos de Guatemala (USAC), fundada el 31 de enero de 1676, es la única universidad estatal en el país. Tiene como función dirigir, organizar y desarrollar la educación superior y elevar el nivel espiritual del país conservando, promoviendo y difundiendo la cultura y el conocimiento científico. Su campus central está ubicado en la Ciudad Universitaria, Zona 12, de la Ciudad de Guatemala.

\vspace{4mm}
\newline
La Escuela de Ciencias Físicas y Matemáticas (ECFM), es la unidad académica de la USAC que en la actualidad brinda las carreras de Física Aplicada y Matemática Aplicada. Asimismo, la ECFM cuenta con el Instituto de Investigación de Ciencias Físicas y Matemáticas (ICFM), unidad que promueve y realiza investigación científica en los campos de Física y Matemática. 

Los principales objetivos de trabajo del ICFM son:
\begin{itemize}
    \item Promoción e investigación académica en ciencia básica y aplicada
    \item Difusión y divulgación del conocimiento generado por la investigación en ciencias físicas y matemáticas.
\end{itemize}

\vspace{4mm}

\section{Descripción del Grupo de Trabajo}
\subsection{Grupo de Investigación en Física de la Materia Condensada}

La Física de materia condensada es un área muy amplia e interesante.  Mediante el estudio de las propiedades  fundamentales de la materia se pueden observar fenómenos interesantes que permiten el desarrollo de nuevas tecnologías.  En años recientes, el conocimiento sobre la relación entre estos fenómenos a nivel subatómico y las propiedades generales de los materiales, ha permitido el desarrollo de materiales con nuevas propiedades o materiales de diseño.

\vspace{4mm}
\newline

Por otro lado, en los últimos años, uno de los desarrollos más emocionantes de la física en los últimos años es la convergencia entre la física de la materia condensada y las áreas de información y computación cuánticas. Esto ha permitido un avance acelerado en ambos campos, logrando ir de la mano teoría y experimento.  El principal motivo de esta convergencia es el estudio de los sistemas cuánticos de muchos cuerpos; donde es indispensable entender el entrelazamiento cuántico. Un recurso escencial para la teoría de la información cuántica.

\vspace{4mm}
\newline

El Grupo de trabajo se enfoca en las áreas de la física de la materia condensada, incluyendo varios subgrupos de estudio aplicados a mecánica estadística, mecánica cuántica, óptica, nanociencia, información cuántica, computación cuántica, entre otros.

\tableofcontents

\bigskip

\newpage
\section{Introducción}

El estudio de la dinámica de sistemas cuánticos abiertos es fundamental para el avance de la computación e información cuántica. Los canales cuánticos describen la evolución de estados cuánticos en  interacción con su entorno, y son clave para el entendimiento de este tipo de sistemas.

En este trabajo, nos enfocamos en la aplicación de algoritmos variacionales cuánticos (VQA) para la inicialización y optimización de canales cuánticos en plataformas de computación cuántica como \textit{Qiskit}\cite{Qiskit}. Los VQA, a través de su enfoque híbrido que combina computación cuántica y optimización clásica, nos ofrecen una técnica óptima para ajustar los parámetros que gobiernan los canales cuánticos, permitiendo una mayor precisión y control en el procesamiento cuántico \cite{VQA}.




\section{Objetivos}
\subsection{Objetivo general}

Estudiar y desarrollar Algoritmos Variacionales Cuánticos para la implementación de canales cuánticos de un  qubit en computadoras cuánticas.


\subsubsection{Objetivos específicos}


\begin{enumerate}
    \item Estudiar la teoría y uso de computadoras cuánticas.
    \item Comprender las definiciones fundamentales de la teoría de sistemas cuánticos abiertos y canales cuánticos.
    \item Investigar el uso de los Algoritmos Variacionales Cuánticos en la resolución de problemas de información cuántica.
    \item Implementar canales cuánticos de un qubit en computadoras cuánticas y analizar su comportamiento.
\end{enumerate}

\section{Justificación de Proyecto}
El estudio y la implementación de canales cuánticos son fundamentales para optimizar las operaciones en computadoras cuánticas, ya que permiten modelar los efectos del ruido y errores en sistemas cuánticos abiertos. La implementación de estos canales puede mejorar la fidelidad en los resultados de los cálculos cuánticos.

Dado que la implementación de canales cuánticos en hardware cuántico actual no es trivial, es necesario desarrollar nuevos métodos que aborden estos retos. En este trabajo proponemos el uso de Algoritmos Variacionales Cuánticos (VQA) como una herramienta eficaz para lograr dicho objetivo. Estos métodos permiten optimizar parámetros para implementar canales cuánticos en computadoras cuánticas reales.

Este proyecto es relevante porque explora enfoques no tratados durante la formación académica básica, enfocándose en áreas avanzadas de la computación cuántica. Por lo tanto, se considera de gran importancia investigar y desarrollar nuevas técnicas que contribuyan a la mejora de la implementación de canales cuánticos. 

Por último, este proyecto contribuirá significativamente a la sociedad del país, donde la investigación científica, especialmente en estos campos, requiere un mayor desarrollo. Este tipo de estudios es importante para la Escuela de Ciencias Físicas y Matemáticas en Guatemala y al Instituto de Física en México, ya que se pueden abrir brechas a investigaciones profundas en el área de la información y computación cuántica.

\section{Marco Teórico}

\subsection{Matriz Densidad}

Un sistema cuántico se describe mediante un vector complejo $\ket{\psi}$, conocido como estado cuántico. En el caso de un estado puro, este se representa como: 
$ \ket{\psi} = \sum_{i=1}^{d} c_i \ket{i}, $
donde $c_i$ son coeficientes complejos y $\{\ket{i}\}$ es una base ortonormal \cite{nielsen_chuang_2011}.

Sin embargo, un sistema cuántico también puede estar en un ensamble estadístico de estados $\{\ket{\psi_i}\}$ con probabilidades o "pesos" $\{p_i\}$. Este tipo de sistema se denomina estado mixto.

Tanto los estados puros como los mixtos se describen mediante un operador conocido como matriz densidad. Para un estado puro, la matriz densidad está dada por:
\begin{equation}
    \label{ec:1.5}
    \rho = \ketbra{\psi}.
\end{equation}
En el caso de un estado mixto, la matriz densidad se expresa como:
\begin{equation}
\label{dens_mixtos}
    \rho = \sum_{i=1}^{n} p_i \ketbra{\psi_i}.
\end{equation}




\subsection{Canales Cuánticos y su representación}
La dinámica de un sistema cuántico abierto se describe mediante la interacción de un sistema principal y un entorno, formando un sistema cuántico cerrado \cite{nielsen_chuang_2011}. Esta interacción se da por una transformación unitaria $U$, resultando en el estado final del sistema principal $\mathcal{E}$:
\begin{equation}
    \mathcal{E}(\rho_{\text{sp}}) = \text{Tr}_{\text{ent}}[ U( \rho_{\text{sp}} \otimes \rho_{\text{ent}} ) U^\dagger ].
\end{equation}

Los operadores cuánticos pueden representarse mediante la \textit{representación de Kraus}, donde los operadores ${E_k}$ satisfacen la \textit{relación de completitud}.  Por otro lado, la matriz de Choi es otra representación útil para caracterizar los canales cuánticos, definida como \cite{Choi2024}:
\begin{equation}
\Lambda_{\mathcal{E}} = (I \otimes \mathcal{E})(|\Omega\rangle\langle\Omega|).
\end{equation}

\subsection{Circuitos Cuánticos}
Un circuito cuántico es una rutina computacional con operadores que actúan sobre qubits. Los circuitos se conforman por cables, compuertas, mediciones y \textit{resets}. Un cable puede representar el paso del tiempo, como una partícula física, en nuestro caso representará a uno o varios qubits en el estado inicial $\ket{0}$ \cite{nielsen_chuang_2011}. 

Las compuertas cuánticas son operadores unitarios reversibles $U$ de dimensión
$n\times n $. Estas actúan linealmente sobre qubits que pueden estar
en superposición. Además, se visualizan por cajas que contienen las
iniciales del operador aplicado (ver figura \ref{fig:compuerta}). \par

  \begin{figure}[h]
        \centering
        \begin{quantikz}
        \qw & \gate{U} & \qw
        \end{quantikz}
        \caption{Representación de compuerta cuántica $U$}
        \label{fig:compuerta}
    \end{figure}

Los operadores $U$ que actúan sobre un qubit son de dimensión $2 \times 2$, y se representan por la matriz
\begin{equation}
    U = \begin{pmatrix}
    u_{00} & u_{01}  \\
    u_{10} & u_{11} 
    \end{pmatrix}.
\end{equation}

Podemos correr circuitos cuánticos en la plataforma de IBM Quantum Experience por medio de simuladores, que emulan de forma cercana el comportamiento de una computadora cuántica, con capacidad de hasta 5000 qubits. También podemos correr los circuitos en computadoras cuánticas reales,  con capacidad de hasta 7 qubits de forma gratuita. En este trabajo haremos uso de los servicios gratuitos, conociendo sus limitaciones.\par 


 \subsection{Algoritmos Variacionales Cuánticos (VQA)}
 
Los VQA son algoritmos que combinan la computación cuántica y los métodos
clásicos de optimización para entrenar un circuito cuántico parametrizado \cite{VQA}. La
solución del problema se codifica en una función de costo $C$, y se define un
\textit{ansatz}, que es una suposición inicial sobre la forma del circuito
cuántico parametrizado por un conjunto de parámetros $\{\theta\}$. Esta
hipótesis asume que el circuito puede generar un estado cuántico $\ket{\phi}$
que minimice la función de costo $C$. El ansatz se entrena posteriormente es
un bucle cuántico-clásico para resolver la tarea de optimización:
\begin{equation}
\theta^* = \text{arg }\underset{\theta}{\text{min}} C (\theta).
\end{equation}


\section{Plan de Trabajo}

El proyecto estará dividido en tres capítulos que cubrirán tanto los aspectos teóricos como la implementación y los resultados obtenidos al trabajar con canales cuánticos utilizando VQA.

En el primer capítulo, se abordarán los fundamentos teóricos necesarios. Se comenzará con una introducción general a la información cuántica y su relevancia para el estudio de los canales cuánticos. Se explicará el concepto de matriz densidad, tanto para estados puros como mixtos, y sus propiedades fundamentales. También se profundizará en la teoría de los canales cuánticos, cómo se representan y qué impacto tienen en los estados cuánticos. Finalmente, se describirán los circuitos cuánticos y su papel en la implementación práctica de estos canales.

En el segundo capítulo, el enfoque será la teoría e implementación de los VQA. Se hará una introducción general sobre los VQA, explicando su papel como método híbrido cuántico-clásico. Luego, se detallará su estructura, desde los ansatz cuánticos hasta las funciones de costo y la optimización clásica. Se hará énfasis en la implementación de canales cuánticos de un qubit utilizando estos algoritmos, describiendo los circuitos necesarios y las rutinas de optimización.

En el tercer capítulo, se presentarán los resultados obtenidos en la implementación de un canal de despolarización para un qubit. Se detallará el algoritmo empleado, discutiendo cómo se optimizaron los parámetros y se implementaron los circuitos. Finalmente, se mostrarán y analizarán los resultados obtenidos, discutiendo su significado, su comparación con estudios previos, y proponiendo posibles mejoras en los algoritmos desarrollados.



\section{Cronograma de Actividades}
El trabajo de este EPS se trabajará un promedio de 4 horas diarias, en horario nocturo de lunes a viernes durante los próximos seis meses. Las tareas necesarias para cada mes se enlistan en la siguiente sección: 

\begin{itemize}
        \item Tarea 1:Investigar de bibliografías, artículos científicos y documentación relevante sobre canales cuánticos y VQA. 
    \item Tarea 2: Revisar y organizar de la información obtenida en la Tarea 1, incluyendo la elaboración de resúmenes y notas que ayuden a entender mejor los conceptos aprendidos. También se identificarán áreas clave que requieran más estudio para el avance del proyecto.
    
    \item Tarea 3: Programar y probar códigos en Python para implementar los canales cuánticos en computadoras cuánticas.
    \item Tarea 4: Analizar e interpretar los datos obtenidos de las pruebas.
    \item Tarea 5: Documentar los resultados en un repositorio en línea para facilitar su revisión y consulta.
    \item Tarea 6: Redactar el informe final del proyecto.
\end{itemize}


\begin{center}
\begingroup
\small % Reduce el tamaño de la fuente
\setlength{\tabcolsep}{4pt} % Reduce el espacio entre columnas
\renewcommand{\arraystretch}{1.0} % Reduce la altura de las filas
\begin{tabular}{|c|c|c|c|c|c|c|}
\hline
\textbf{Tareas} & \textbf{Octubre 2024} & \textbf{Noviembre 2024} & \textbf{Diciembre 2024} & \textbf{Enero 2025} & \textbf{Febrero 2025} & \textbf{Marzo 2025}\\ \hline
Tarea 1 & \cellcolor[gray]{0.5} &  & & & & \\ \hline
Tarea 2 & \cellcolor[gray]{0.5} & \cellcolor[gray]{0.5} & & & & \\ \hline
Tarea 3 & & & \cellcolor[gray]{0.5} & & & \\ \hline
Tarea 4 & & & \cellcolor[gray]{0.5} & \cellcolor[gray]{0.5} & & \\ \hline
Tarea 5 & & & \cellcolor[gray]{0.5} & \cellcolor[gray]{0.5} & & \\ \hline
Tarea 6 & & & &  \cellcolor[gray]{0.5}& \cellcolor[gray]{0.5} & \cellcolor[gray]{0.5} \\ \hline
\end{tabular}
\endgroup
\end{center}



\bibliographystyle{unsrt}
\bibliography{biblio}

\end{document}