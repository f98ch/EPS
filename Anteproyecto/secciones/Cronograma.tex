\section{Cronograma de Actividades}
El trabajo de este EPS se trabajará un promedio de 4 horas diarias, en horario nocturo de lunes a viernes durante los próximos seis meses. Las tareas necesarias para cada mes se enlistan en la siguiente sección: 

\begin{itemize}
        \item Tarea 1:Investigar de bibliografías, artículos científicos y documentación relevante sobre canales cuánticos y VQA. 
    \item Tarea 2: Revisar y organizar de la información obtenida en la Tarea 1, incluyendo la elaboración de resúmenes y notas que ayuden a entender mejor los conceptos aprendidos. También se identificarán áreas clave que requieran más estudio para el avance del proyecto.
    
    \item Tarea 3: Programar y probar códigos en Python para implementar los canales cuánticos en computadoras cuánticas.
    \item Tarea 4: Analizar e interpretar los datos obtenidos de las pruebas.
    \item Tarea 5: Documentar los resultados en un repositorio en línea para facilitar su revisión y consulta.
    \item Tarea 6: Redactar el informe final del proyecto.
\end{itemize}


\begin{center}
\begingroup
\small % Reduce el tamaño de la fuente
\setlength{\tabcolsep}{4pt} % Reduce el espacio entre columnas
\renewcommand{\arraystretch}{1.0} % Reduce la altura de las filas
\begin{tabular}{|c|c|c|c|c|c|c|}
\hline
\textbf{Tareas} & \textbf{Octubre 2024} & \textbf{Noviembre 2024} & \textbf{Diciembre 2024} & \textbf{Enero 2025} & \textbf{Febrero 2025} & \textbf{Marzo 2025}\\ \hline
Tarea 1 & \cellcolor[gray]{0.5} &  & & & & \\ \hline
Tarea 2 & \cellcolor[gray]{0.5} & \cellcolor[gray]{0.5} & & & & \\ \hline
Tarea 3 & & & \cellcolor[gray]{0.5} & & & \\ \hline
Tarea 4 & & & \cellcolor[gray]{0.5} & \cellcolor[gray]{0.5} & & \\ \hline
Tarea 5 & & & \cellcolor[gray]{0.5} & \cellcolor[gray]{0.5} & & \\ \hline
Tarea 6 & & & &  \cellcolor[gray]{0.5}& \cellcolor[gray]{0.5} & \cellcolor[gray]{0.5} \\ \hline
\end{tabular}
\endgroup
\end{center}
