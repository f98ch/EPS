\section{Descripción general de la institución}

La Universidad de San Carlos de Guatemala (USAC), fundada el 31 de enero de 1676, es la única universidad estatal en el país. Tiene como función dirigir, organizar y desarrollar la educación superior y elevar el nivel espiritual del país conservando, promoviendo y difundiendo la cultura y el conocimiento científico. Su campus central está ubicado en la Ciudad Universitaria, Zona 12, de la Ciudad de Guatemala.

\vspace{4mm}
\newline
La Escuela de Ciencias Físicas y Matemáticas (ECFM), es la unidad académica de la USAC que en la actualidad brinda las carreras de Física Aplicada y Matemática Aplicada. Asimismo, la ECFM cuenta con el Instituto de Investigación de Ciencias Físicas y Matemáticas (ICFM), unidad que promueve y realiza investigación científica en los campos de Física y Matemática. 

Los principales objetivos de trabajo del ICFM son:
\begin{itemize}
    \item Promoción e investigación académica en ciencia básica y aplicada
    \item Difusión y divulgación del conocimiento generado por la investigación en ciencias físicas y matemáticas.
\end{itemize}

\vspace{4mm}

\section{Descripción del Grupo de Trabajo}
\subsection{Grupo de Investigación en Física de la Materia Condensada}

La Física de materia condensada es un área muy amplia e interesante.  Mediante el estudio de las propiedades  fundamentales de la materia se pueden observar fenómenos interesantes que permiten el desarrollo de nuevas tecnologías.  En años recientes, el conocimiento sobre la relación entre estos fenómenos a nivel subatómico y las propiedades generales de los materiales, ha permitido el desarrollo de materiales con nuevas propiedades o materiales de diseño.

\vspace{4mm}
\newline

Por otro lado, en los últimos años, uno de los desarrollos más emocionantes de la física en los últimos años es la convergencia entre la física de la materia condensada y las áreas de información y computación cuánticas. Esto ha permitido un avance acelerado en ambos campos, logrando ir de la mano teoría y experimento.  El principal motivo de esta convergencia es el estudio de los sistemas cuánticos de muchos cuerpos; donde es indispensable entender el entrelazamiento cuántico. Un recurso escencial para la teoría de la información cuántica.

\vspace{4mm}
\newline

El Grupo de trabajo se enfoca en las áreas de la física de la materia condensada, incluyendo varios subgrupos de estudio aplicados a mecánica estadística, mecánica cuántica, óptica, nanociencia, información cuántica, computación cuántica, entre otros.