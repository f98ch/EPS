\section{Justificación de Proyecto}
El estudio y la implementación de canales cuánticos son fundamentales para optimizar las operaciones en computadoras cuánticas, ya que permiten modelar los efectos del ruido y errores en sistemas cuánticos abiertos. La implementación de estos canales puede mejorar la fidelidad en los resultados de los cálculos cuánticos.

Dado que la implementación de canales cuánticos en hardware cuántico actual no es trivial, es necesario desarrollar nuevos métodos que aborden estos retos. En este trabajo proponemos el uso de Algoritmos Variacionales Cuánticos (VQA) como una herramienta eficaz para lograr dicho objetivo. Estos métodos permiten optimizar parámetros para implementar canales cuánticos en computadoras cuánticas reales.

Este proyecto es relevante porque explora enfoques no tratados durante la formación académica básica, enfocándose en áreas avanzadas de la computación cuántica. Por lo tanto, se considera de gran importancia investigar y desarrollar nuevas técnicas que contribuyan a la mejora de la implementación de canales cuánticos. 

Por último, este proyecto contribuirá significativamente a la sociedad del país, donde la investigación científica, especialmente en estos campos, requiere un mayor desarrollo. Este tipo de estudios es importante para la Escuela de Ciencias Físicas y Matemáticas en Guatemala y al Instituto de Física en México, ya que se pueden abrir brechas a investigaciones profundas en el área de la información y computación cuántica.