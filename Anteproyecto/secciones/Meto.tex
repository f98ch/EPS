\section{Plan de Trabajo}

El proyecto estará dividido en tres capítulos que cubrirán tanto los aspectos teóricos como la implementación y los resultados obtenidos al trabajar con canales cuánticos utilizando VQA.

En el primer capítulo, se abordarán los fundamentos teóricos necesarios. Se comenzará con una introducción general a la información cuántica y su relevancia para el estudio de los canales cuánticos. Se explicará el concepto de matriz densidad, tanto para estados puros como mixtos, y sus propiedades fundamentales. También se profundizará en la teoría de los canales cuánticos, cómo se representan y qué impacto tienen en los estados cuánticos. Finalmente, se describirán los circuitos cuánticos y su papel en la implementación práctica de estos canales.

En el segundo capítulo, el enfoque será la teoría e implementación de los VQA. Se hará una introducción general sobre los VQA, explicando su papel como método híbrido cuántico-clásico. Luego, se detallará su estructura, desde los ansatz cuánticos hasta las funciones de costo y la optimización clásica. Se hará énfasis en la implementación de canales cuánticos de un qubit utilizando estos algoritmos, describiendo los circuitos necesarios y las rutinas de optimización.

En el tercer capítulo, se presentarán los resultados obtenidos en la implementación de un canal de despolarización para un qubit, utilizando los VQA. Se detallará el algoritmo empleado para un canal despolarizante, discutiendo cómo se optimizaron los parámetros y se implementaron los circuitos. Finalmente, se mostrarán y analizarán los resultados obtenidos, discutiendo su significado, su comparación con estudios previos, y proponiendo posibles mejoras en los algoritmos desarrollados.

