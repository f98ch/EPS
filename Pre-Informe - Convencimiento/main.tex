\documentclass[11pt, spanish, letterpage]{report} % {{{
\usepackage[T1]{fontenc}
\usepackage[utf8]{inputenc}
\usepackage[letterpaper]{geometry}
\geometry{verbose,tmargin=2cm,bmargin=2.5cm,lmargin=2cm,rmargin=2cm}
%\pagestyle{plain}
\setlength{\parskip}{\baselineskip}
\setlength{\parindent}{4mm}
\usepackage{graphicx}
%\usepackage{setspace}
\usepackage{tabulary}
\usepackage{amsmath}
\usepackage{amsfonts}
\usepackage{amssymb}
\usepackage{amsthm}
\usepackage{physics}
\usepackage{wrapfig}
\usepackage{bbold}
\usepackage{tikz}
\usetikzlibrary{quantikz}

\usepackage{fancybox}
\usepackage{colortbl}
\usepackage{amsbsy}
\usepackage[draft,inline,nomargin]{fixme} \fxsetup{theme=color}

%-----------------------------------------
\usepackage[numbers]{natbib} %agregué [numbers] para que no haya problema al usar \bibliographystyle{naturemag} (este estilo incluye los url al citar páginas web)
%\setcitestyle{authoryear,open={(},close={)}} % para citar las páginas
\usepackage{bbold}

\definecolor{mycolor}{RGB}{24,128,48}

\usepackage{physics}
\usepackage{fancybox}
\usepackage{colortbl}
\usepackage{amsbsy}
\usepackage[draft,inline,nomargin]{fixme} \fxsetup{theme=color}
\FXRegisterAuthor{cp}{acp}{\color{blue}CP}
\FXRegisterAuthor{fel}{afel}{\color{mycolor}F}
\FXRegisterAuthor{sub}{asub}{\color{red}Sub}

\usepackage{graphicx}
\graphicspath{ {./img/} }


\usepackage[]{lineno} 
% \setlength\linenumbersep{3pt}

\newcommand{\fref}[1]{fig.\ref{#1}}   \newcommand{\tref}[1]{table\ref{#1}}
\newcommand{\Fref}[1]{Fig.\ref{#1}}  \newcommand{\Tref}[1]{Table\ref{#1}}
\newcommand{\Cref}[1]{Cuadro~\ref{#1}}

\newcommand{\psii}{\psi_i}
\newcommand{\Pk}[1]{\ket{\psi_{#1} }}
\newcommand{\Pb}[1]{\bra{\psi_{#1} }}
\newcommand{\pk}{\ket{\psi}}
\newcommand{\M}{\mathcal{M}^{(N)}}
\newcommand{\E}{\mathcal{E}}
\newcommand{\Erho}{\mathcal{E}(\rho)}
\newcommand{\1}{\mathbb{1}}
\newcommand{\ten}{\otimes}
\newcommand{\h}[1]{\colorbox{yellow}{#1}}
\newcommand{\hi}{\mathcal{H}}
\newcommand{\txt}[1]{\text{#1}}
\newcommand{\here}{\h{\hspace{15cm}} }
\newcommand{\rhoi}{\dyad{\psii}{\psii}}
\newcommand{\ind}[2]{{{}^{#1}_{#2}}}
\newcommand{\rc}[1]{r_{#1}}
\newcommand{\pauli}[2]{\sigma_{#1}\otimes\sigma_{#2}}
\newcommand{\esqueleto}[1]{\textcolor{mycolor}{#1}}
\newcommand{\ot}{\otimes}
\newcommand{\m}{\textcolor{mycolor}{|}}

% Para que funcione mejor la numeración {{{
% https://tex.stackexchange.com/questions/43648/why-doesnt-lineno-number-a-paragraph-when-it-is-followed-by-an-align-equation
\newcommand*\patchAmsMathEnvironmentForLineno[1]{%
  \expandafter\let\csname old#1\expandafter\endcsname\csname #1\endcsname
  \expandafter\let\csname oldend#1\expandafter\endcsname\csname end#1\endcsname
  \renewenvironment{#1}%
     {\linenomath\csname old#1\endcsname}%
     {\csname oldend#1\endcsname\endlinenomath}}% 
\newcommand*\patchBothAmsMathEnvironmentsForLineno[1]{%
  \patchAmsMathEnvironmentForLineno{#1}%
  \patchAmsMathEnvironmentForLineno{#1*}}%
\AtBeginDocument{%
\patchBothAmsMathEnvironmentsForLineno{equation}%
\patchBothAmsMathEnvironmentsForLineno{align}%
\patchBothAmsMathEnvironmentsForLineno{flalign}%
\patchBothAmsMathEnvironmentsForLineno{alignat}%
\patchBothAmsMathEnvironmentsForLineno{gather}%
\patchBothAmsMathEnvironmentsForLineno{multline}%


}
% }}}


\usepackage{lipsum}
\usepackage{babel}
\usepackage{multirow}
\usepackage{array}
\newtheorem{ex}{Ejemplo}[section]


%Para dibujar circuitos cuánticos. 
\usepackage{tikz}
\usetikzlibrary{quantikz}


\usepackage[]{lineno}  \linenumbers
\setlength\linenumbersep{3pt}

\renewcommand{\baselinestretch}{1} % interlinado
\addto\shorthandsspanish{\spanishdeactivate{~<>}}
\date{}
\spanishdecimal{.}
\usepackage{multicol}%para escribir en muchas columnas
%para que no corte palabras
\usepackage[none]{hyphenat}
\usepackage{times}
%\onehalfspacing

\usepackage{hyperref}
%\usepackage{biblatex}
%\addbibresource{references.bib}

%---ejercicios, problemas -teoremas
%---Problemas encerrados-Bonitos
\usepackage[framemethod=tikz]{mdframed}
\mdfsetup{skipabove=\topskip,skipbelow=\topskip}
\newcounter{problem}[section]
\newenvironment{problem}[1][]{%
%\stepcounter{problem}%
\ifstrempty{#1}%
{\mdfsetup{%
frametitle={%
	\tikz[baseline=(current bounding box.east),outer sep=0pt]
	\node[anchor=east,rectangle,fill=brown!50]
{\strut Problema~\theproblem};}}

}%
{\mdfsetup{%
frametitle={%
	\tikz[baseline=(current bounding box.east),outer sep=0pt]
	\node[anchor=east,rectangle,fill=brown!50]
{\strut Problema ~\theproblem:~#1};}}%

}%
\mdfsetup{innertopmargin=5pt,linecolor=black!50,%
	linewidth=2pt,topline=true,
	frametitleaboveskip=\dimexpr-\ht\strutbox\relax,}
	
\begin{mdframed}[]\relax%
}{\end{mdframed}}

\newenvironment{solution}% environment name
{\colorbox{gray}{~~\textbf{\textcolor{white}{Solución:}}~~}~~}%
{}
%-----end------------

%\newtheorem{example}{Ejemplo}[chapter]
\newtheorem{teorema}{Teorema}[chapter]
%%---

\newcommand{\Ev}{\mathbf{E}}
\newcommand{\rv}{\mathbf{r}}
\newcommand{\ru}{\hat{\rv}}
\usepackage{tabulary}
%---paquetes para fisicaz
\usepackage{physics}%facilita la escritura de operadores usados en fisica
%-paquete para unidades en el sistema internacional
% \usepackage[load=prefix, load=abbr, load=physical]{siunitx}
% \newunit{\gram}{g }%gramos
% \newunit{\velocity}{ \metre / \Sec }%unidades de velocidad sistema internacional
% \newunit{\acceleration}{ \metre / \Sec^2 }%unidades de aceleracion sistema internacional
% \newunit{\entropy}{ \joule / \kelvin }%unidades de entropia sisteme internacional
%--definiendo constantes fisicas en el SI
\newcommand{\accgravity}{9.8 \metre / \Sec^2}
%---diferencial inexacta
\newcommand{\dbar}{\mathchar'26\mkern-12mu d}

\oddsidemargin 0in
\textwidth 6.5in
\topmargin -0.5in
\textheight 8.5in
% }}}
\begin{document}
\begin{titlepage} % {{{ Suppresses displaying the page number on the title page and the subsequent page counts as page 1                                  
\newcommand{\HRule}{\rule{\linewidth}{0.5mm}} % Defines a new command for horizontal lines, change thickness here                             

\center % Centre everything on the page                                                                                                       

%------------------------------------------------
%       Title
%------------------------------------------------
\HRule \\[0.4cm] % Adjusted space before and after the rule
{\huge\bfseries Implementación de Canales Cuánticos de 1 qubit con VQA\\[0.3cm]} % Adjusted space before and after the title
\HRule \\[2cm] % Adjusted space before and after the rule
% \end{titlepage}

%------------------------------------------------                                                                                             
%       Author(s)                                                                                                                             
%------------------------------------------------                                                                                             


\Large{\textbf{Felipe Antonio Ixcamparic Choy}}\\[2cm] % Your name

%------------------------------------------------
%       Headings
%------------------------------------------------

\textsc{\LARGE Universidad de San Carlos de Guatemala\\ Escuela de Ciencias Físicas y Matemáticas\\ Licenciatura en Física}\\[2cm]

\textsc{\Large Resumen de trabajo de EPS - Implementación de Canales Cuánticos mediante Algoritmos Variacionales Cuánticos}\\[2cm]
\textsc{\Large Supervisado por: \textbf{Dr. Carlos Pineda (IF-UNAM) y\\M.Sc. Juan Diego Chang (ECFM-USAC)}}
                                                                                                      

%------------------------------------------------                                                                                             
%       Date                                                                                                                                  
%------------------------------------------------                                                                                             
\vfill\vfill\vfill % Position the date 3/4 down the remaining page
\vfill\vfill\vfill

% {\large 19 de noviembre de 2021} % Date, change the \today to a set date if you want to be precise                                                              

%------------------------------------------------                                                                                             
%       Logo                                                                                                                                  
%------------------------------------------------                                                                                             

%----------------------------------------------------------------------------------------                                                     

\vfill % Push the date up 1/4 of the remaining page                                                                                           

\end{titlepage} % }}}
% Inicio % {{{
\section{Objetivo General} % {{{
Estudiar  y desarrollar algoritmos variacionales cuánticos para la implementación de canales cuánticos de un \textit{qubit} en computadoras cuánticas.
% }}}
\section{Objetivos Específicos} % {{{

\begin{itemize}
\item Estudiar la teoría y uso de computadoras cuánticas.


    \item Comprender las definiciones fundamentales de la teoría de sistemas cuánticos abiertos y canales cuánticos.
    
    \item Investigar el uso de los Algoritmos Variacionales Cuánticos en la resolución de problemas de información cuántica.

    \item Impelemntar Canales cuánticos en computadoras cuánticas y analizar su comportamiento.
\end{itemize}


% }}}
\section{Introducción}

El estudio de la dinámica de sistemas cuánticos abiertos es fundamental para el avance de la computación e información cuántica. Los canales cuánticos describen la evolución de estados cuánticos en  interacción con su entorno, y son clave para el entendimiento de este tipo de sistemas.

En este trabajo, nos enfocamos en la aplicación de Algoritmos Variacionales Cuánticos (VQA) para la inicialización y optimización de canales cuánticos en plataformas de computación cuántica. Los VQA, a través de su enfoque híbrido que combina computación cuántica y optimización clásica, nos ofrecen una técnica óptima para ajustar los parámetros que gobiernan los canales cuánticos, permitiendo una mayor precisión y control en el procesamiento cuántico.
\chapter{Fundamentos Teóricos}
*Capítulo a heredar de informe de prácticas. 
\section{Introducción}
\section{Matriz Densidad}
*Sección heredada de informe de prácticas

Un sistema cuántico se describe mediante un vector complejo $\ket{\psi}$, conocido como estado cuántico. Un estado cuántico puro se representa como:
$
\ket{\psi} = \sum_{i=1}^{d} c_i \ket{i},
$
donde $c_i$ son coeficientes complejos y $\{\ket{i}\}$ es una base ortonormal. Los coeficientes deben cumplir con la condición de normalización:
$
\braket{\psi}{\psi}=1.
$

 Los estados puros pueden representarse mediante la matriz densidad:$\rho = \ketbra{\psi}.$.  
 
 Un estado mixto es un ensamble estadístico de estados puros $\{\ket{\psi_i}\}$ con probabilidades $\{p_i\}$. La matriz densidad para estados mixtos es:$\rho = \sum_{i=1}^{n} p_i \ketbra{\psi_i}.$

La matriz densidad $\rho$ tiene tres propiedades principales: Hermiticidad , traza unitaria y no negatividad.

La pureza de un estado cuántico mide qué tan mixto es el estado. Se define como $\text{Pureza} = tr(\rho^2).$

La traza parcial se usa para obtener la matriz densidad reducida de un subsistema en estados compuestos. Por ejemplo, si tenemos un sistema compuesto por dos subsistemas $A$ y $B$, la matriz densidad reducida del subsistema $A$ se obtiene trazando sobre el subsistema $B$:
$\rho_A = tr_B(\rho_{AB}).$

La purificación es un proceso para representar un estado mixto como un estado puro en un espacio de Hilbert ampliado. Esto es útil en computadoras cuánticas para iniciar estados mixtos.



\section{Canales Cuánticos y su representación}
La dinámica de un sistema cuántico abierto se describe mediante la interacción de un sistema principal y un entorno, formando un sistema cuántico cerrado \cite{nielsen_chuang_2011}. Esta interacción se da por una transformación unitaria $U$, resultando en el estado final del sistema principal $\mathcal{E}$.

Los operadores cuánticos pueden representarse mediante la \textit{representación de Kraus}, donde los operadores ${E_k}$ satisfacen la \textit{relación de completitud}. La \textit{matriz de Choi} es otra representación útil para caracterizar los canales cuánticos, definida como $\Lambda_{\mathcal{E}} = (I \otimes \mathcal{E})(|\Omega\rangle\langle\Omega|)$.

La matriz de Choi tiene varias propiedades: es Hermítica, puede expresarse en términos de los operadores de Kraus, es positivo semidefinido y su traza parcial es la identidad si preserva la traza. Además, la matriz de Choi de un operador CPTP es única, mientras que la representación de Kraus no lo es.

El \textit{Depolarizing Channel} es un ejemplo de canal cuántico que introduce ruido aleatorio en un sistema. Su matriz de Choi refleja cómo el canal mezcla el estado original con la matriz identidad, reduciendo la pureza del estado.

En los siguientes capítulos se detallará el proceso para implementar la matriz de Choi y reconstruirla en sistemas cuánticos.
\section{Circuitos Cuánticos}
*Heredado de informe de prácticas 

Previo a hablar sobre compuertas y circuitos cuánticos, introducimos la Esfera de Bloch para representar geométricamente al qubit. Esta esfera nos permite visualizar los cambios del estado de un qubit bajo transformaciones unitarias $U$. 

Las computadoras cuánticas tienen compuertas cuánticas, operadores unitarios reversibles $U$ que actúan linealmente sobre qubits en superposición. Las compuertas $X$, $Y$ y $Z$ representan las matrices de Pauli $\sigma_x$, $\sigma_y$ y $\sigma_z$ respectivamente.

Un circuito cuántico es una rutina computacional con operadores que actúan sobre qubits. Los circuitos se conforman por cables, compuertas y mediciones.

Para este trabajo, utilizamos la plataforma \textit{IBM Quantum Experience}, que nos da acceso a computadoras cuánticas para ejecutar circuitos cuánticos válidos. 

\chapter{Algoritmos Variacionales Cuánticos}
\section{Introducción}
\section{Descripción de los VQA}
Los Algoritmos Variacionales Cuánticos son una clase de algoritmos que combinan computación cuántica y clásica para resolver problemas de optimización. 

El ansatz es una suposición inicial o una forma parametrizada del estado cuántico que se va a optimizar. La calidad del ansatz afecta directamente la eficiencia del VQA.

La función de costo $C(\theta)$ mide la calidad de la solución propuesta por el algoritmo cuántico. Esta función es evaluada en un circuito cuántico y luego optimizada clásicamente.

La optimización de la función de costo se realiza mediante métodos clásicos, como el descenso de gradientes, que ajustan los parámetros del ansatz para minimizar la función de costo.

El método de gradientes calcula la derivada de la función de costo respecto a los parámetros del ansatz y ajusta estos parámetros para encontrar el mínimo de la función de costo.

\section{Implementación de canales de un qubit usando VQA}
Podemos iniciar un canal cuántico en una computadora cuántica si conocemos el conjunto de compuertas necesarias que repliquen la acción del canal sobre el sistema principal $\rho$.

Hemos propuesto utilizar los VQA para inicializar canales cuánticos de 1 qubit, para esto , nuestra función de costo a optimizar será la Fidelidad entre la matriz de Choi teórica contra la reconstruida en una computadora cuántica.

Esta implementación se realizó en el lenguaje de \textit{Python} por medio de la plataforma de \textit{IBM Quantum Experience}, donde explicaremos el ansatz, así como la metodología para la optimización de la función de costo $C(\theta)$.


\chapter{Resultados para canales de un qubit}
\section{Introducción}
\section{Resultados}
Para nuestro estudio, hemos implementado el \textit{Depolarizing Channel} de un qubit, haciendo una optimización para sus distintos valores de $p$. Esto lo ejecutamos tanto en un simulador como en una computadora cuántica.

Para el análisis de nuestros resultados estudiamos la fidelidad de proceso, pureza y comparación de autovalores y autovectores entre las matrices de Choi. 
\section{Discusión de Resultados}
Como parte de la discusión de resultados analizamos las causas de fidelidades inferiores en la computadora cuántica a diferencia del simulador cuántico. 
\chapter{Conclusiones}

\chapter{Trabajo a Futuro}
\section{Implementación de VQA para 2 qubits}
% En la teoría cuántica, un sistema cuántico abierto es aquel que no está aislado, sino que interactúa continuamente con su entorno. Este tipo de sistemas son fundamentales para entender muchos fenómenos cuánticos en la práctica, ya que cualquier sistema cuántico real interactúa con su entorno. A diferencia de los sistemas cuánticos cerrados, que se consideran completamente aislados y evolucionan de manera unitaria según la ecuación de Schrödinger, los sistemas cuánticos abiertos pueden sufrir pérdidas de coherencia cuántica debido a estas interacciones. En la sección 3.1 exploraremos este tipo de sistemas por medio de canales cuánticos, con énfasis en su representación con la matriz de Choi. En la sección 3.2 mencionaremos la tomografía y fidelidad de proceso cuántico, ya que son importantes para la reconstrucción de canales cuánticos en computadoras cuánticas


% % Pendiente de redactar
% Los Algoritmos Variacionales Cuánticos (VQA) representan una estrategia innovadora en el campo de la computación cuántica para resolver problemas complejos. A través de un esquema híbrido que aprovecha tanto recursos cuánticos como clásicos, los VQA buscan optimizar un conjunto de parámetros en un circuito cuántico parametrizado. Estos algoritmos son fundamentales en tareas donde la solución se codifica dentro de una función de costo o pérdida, la cual es minimizada a través del ajuste variacional de parámetros dentro del circuito. \cite{VQA}

% El circuito cuántico parametrizado, o \textit{ansatz}, es diseñado para ser versátil y adaptarse a la naturaleza del problema y las características del hardware cuántico disponible. Los ansatz más efectivos, como el Hardware Efficient Ansatz (HEA), son aquellos que equilibran la expresividad del circuito y la eficiencia en términos de recursos y operaciones cuánticas requeridas. \cite{VQA}

% El proceso de optimización en VQAs se articula en tres fases principales:
% \begin{enumerate}
%     \item \textbf{Inicialización:} Se comienza con la preparación de un estado cuántico de referencia, sobre el cual se aplicarán las operaciones del \textit{ansatz}.
%     \item \textbf{Aplicación del Ansatz:} Se utiliza el \textit{ansatz} para transformar el estado inicial en un estado final que depende de los parámetros variacionales. Este paso es crítico y requiere una cuidadosa selección y configuración de las puertas cuánticas para navegar el espacio de Hilbert eficazmente.
%     \item \textbf{Evaluación y Optimización:} Se evalúa la función de costo mediante mediciones cuánticas y se emplea un optimizador clásico para ajustar los parámetros en busca de un mínimo, completando así el bucle cuántico-clásico.
% \end{enumerate}
% \subsubsection{Ejemplo: Inicialización de estados mixtos en computadoras cuánticas}
% Faltante de redactar




% \section{Aplicación de VQA para la simulación de Canales Cuánticos}
% Faltante de Redactar





% }}}
\bibliographystyle{abbrvnat}
\bibliography{references}
\end{document}